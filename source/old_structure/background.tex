% !TeX spellcheck = en_US
\chapter{Background}\label{ch:background}]

This chapter presents the background knowledge and the related works to understand the content of the thesis. In \cref{sec:background:mec}, \gls{mec} and its development are introduced. The camera-based perception methods are introduced in \cref{sec:background:robotic_perception}. Moreover, the frameworks and software tools are used to implement the offloading pipeline in this thesis are briefly introduced in \cref{sec:background:frameworks}. Finally, the related works in computation offloading are presented and discussed in \cref{sec:background:related_works}.

% \section{Autonomous Mobile Robot}
% \subsection{Development}
% \subsection{Perception}
% \subsection{SLAM}
% \subsection{Path Planning}
% \subsection{ROS}
% \subsection{Turtlebot}

\section{Multi-access Edge Computing}\label{sec:background:mec}
% could mention TSN (time sensitive network)

% "The goal of \gls{mec} is to provide ultra-low latency and high-bandwidth computing environments for latency-sensitive applications" \cite{Lin2019}
\subsection{Cloud Computing}
\subsection{Fog Computing}
\subsection{Edge Computing and IoT}
\subsection{Wireless TSN}

\section{Camera-based Robotic Perception}\label{sec:background:robotic_perception}
\subsection{CNN}
\subsection{Object detection}
\subsection{Segmentation}
\subsection{YOLO}

\section{Frameworks}\label{sec:background:frameworks}
\subsection{ROS}
% Include a paragraph for qos settings

% Include a paragraph for different rmws
% \gls{ros} is implemented with \gls{udp} for both reliable and best effort \gls{qos}.
% write about ros topics, services, and actions here

% write about packages like nav2, etc.
\subsection{Gazebo}
% write about the difference between ignition and gazebo classic, define the term of gazebo for the rest of the thesis
\subsection{Docker}

\subsection{NetEm} 


\section{Related Works}\label{sec:background:related_works}
\subsection{Purposes}
\subsection{Metrics}
\subsection{Decision-making-based Strategies}
\subsection{Learning-based Strategies}
