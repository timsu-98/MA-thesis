% !TeX spellcheck = en_US
\chapter{Introduction}

This chapter first gives the context of the thesis and introduces the research problem the thesis addresses. Furthermore, this chapter discusses the research questions the thesis is trying to answer and what approaches the thesis adopts. In addition, this chapter also gives a short introduction to the structure of the thesis.

\section{Background}

\glspl{amr} have gained enormous significance in highly automated factories. Rencent developments in robotics and \gls{ai} have enabled the \gls{amr} to operate with autonomy. This requires the \glspl{amr} to run advanced computation- and memory-intensive algorithms related to perception, navigation and learning \cite{Saeik2021, B}
Recent developments in robotics and \gls{ai} have enabled the \glspl{amr} to operate with autonomy. This requires the \glspl{amr} to run computation and memory intensive algorithms related to image processing, path planning, \gls{slam}, and learning \cite{Saeik2021}. However, this also poses a challenge for the robot's on-board system, which has limited computational resources due to spatial and thermal restrictions. Furthermore, with important tasks failing on the \glspl{amr}, it could also cause safety issues for other \glspl{amr} and collaborating humans.

The emergence 

However, it also poses a challenge for the robot's on-board system to run many tasks with the limited computational resources. The emergence of the \gls{mec} provides a solution to this conundrum. 

\section{Research Problem}

\section{Research Question}

\section{Methodology}