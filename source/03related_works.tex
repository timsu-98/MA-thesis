\chapter{Related Work}\label{ch:related_works}

A plethora of research has investigated offloading strategies of edge computing. Offloading strategies decide when and how the \glspl{amr} should offload the computation tasks to the edge. They can be mainly divided into three categories: optimization approaches, game theory approaches, and learning-based approaches. This chapter gives a short introduction to these approaches. 

% Old structure here
% --------------------------------------------------------------
% \section{Task Offloading Goals}\label{sec:task_offloading_goals}

% % Introduce why the robot offloads to edge

% \subsection{Performance improvement}

% % First paragraph, reduce latency

% % Second paragraph, improve computation capabilities

% \subsection{Resource usage}

% \section{Approaches}\label{sec:approaches}

% \subsection{Optimization approaches}

% \subsection{Game theory approaches}

% \subsection{Learning-based approaches}
% --------------------------------------------------------------

\section{Optimization Approaches}\label{sec:optimization_approaches}

Optimization approaches formulate the computation offloading problem as an optimization problem with the goal to minimize energy consumption or execution latency. The multi-robot computation offloading can be usually formulated as an integer linear programming problem. The offloading strategy is derived from the problem solution. 

\citeauthor*{Zhao2015} \cite{Zhao2015} formulate the problem as a joint optimization problem of the radio and computation resources aiming to reduce the overall energy consumption of mobile devices. The energy consumption model consists of two components: energy to transmit data and energy to compute the task locally. \citeauthor*{Zhao2015} also consider task partitioning in computation offloading. In the mathematical model, the computation task can be partitioned and partially offloaded without overhead for partitioning. \citeauthor*{Zhao2015} propose a heuristic strategy to find the sub-optimal solution. \citeauthor*{Chen2015} \cite{Chen2015} formulate the problem as a non-convex quadratic program with the goal of minimizing an offloading cost which consists of weighted energy consumption and execution latency. \citeauthor*{Chen2015} solve the problem by applying a semi-definite relaxation and a randomization mapping method to the optimization problem. It is worth mentioning that the aforementioned works still focus on the mobile cloud computing paradigm. However, the approaches are transferable to edge computing. 

With the emergence of edge computation offloading, edge computers can also be used to perform computational tasks. \citeauthor*{Guo2018} \cite{Guo2018} consider the case where the end devices can choose to offload either to centralized data centers or to edge computers. The problem is formulated as an optimization problem with the goal of minimizing energy consumption with a constraint for maximum execution latency. An exhaustive algorithm is proposed to find the optimal solution. \citeauthor*{Ning2019} \cite{Ning2019} consider not only the three computation options but also the computation task partitioning and distribution among the local system, edge computer, and the cloud. Therefore, the problem is formulated as a mixed integer linear programming problem aiming to reduce the execution latency, which also inspired the problem formulation of the work of this thesis. An iterative heuristic algorithm is proposed to solve the problem. 

The formulated optimization problem with a joint optimization goal is, in general, a mixed integer linear programming problem and cannot be solved. Therefore, on the one hand, heuristic algorithms are proposed to find the suboptimal solution. On the other hand, relaxation is applied to the problem to find the approximate optimal solution. Furthermore, the optimization approaches assume a central decision-making unit that has full knowledge of the entire system. However, with dynamic network conditions and time-sensitive computation tasks, this may not be the case for the edge and the robot. Therefore, an offloading strategy with the ability to adapt to dynamic network changes and satisfy real-time requirements of the computation task is needed. 

\section{Game Theory Approaches}

Unlike the optimization approaches, the game theory approaches formulate the problem as a game where end devices are players of the game. The offloading decision is made by finding an optimum or a \gls{ne} of the game. A \gls{ne} is a group of strategies where no player can profit by modifying his strategy while the other players’ strategies are kept unchanged \cite{Chaari2022}. Therefore, the game theory approaches are intrinsically distributed since the players make their own decisions. 

\citeauthor*{Chen2016} \cite{Chen2016} formulate the distributed computation offloading decision-making problem among mobile device users as a multi-user computation offloading game. The approach aims to minimize a cost function consisting of energy consumption and execution latency. In the end, a distributed computation offloading algorithm is proposed to solve the problem. Furthermore, \citeauthor*{Pham2018} \cite{Pham2018} also consider the scenario where multiple users offload to multiple servers. The problem is formulated as a joint optimization problem with the goal of optimizing the transmit power of the users and the computation resources at the servers. Two distributed matching algorithms are proposed to decide the server and the sub-channel. \citeauthor*{Hong2019} \cite{Hong2019} take another step further by considering the computation offloading routing. The problem is formulated as a multi-hop cooperative computation offloading game. They further proposed a distributed algorithm to attain the \gls{ne} of the game for partitioned and unpartitioned tasks. In addition, \citeauthor*{Xu2020} \cite{Xu2020} investigate how the weight factor in a cost function consisting of different optimization goals affects the overall system performance. 

The game theory approaches allow distributed decision-making. However, the algorithms usually require several iterations for the cost function to converge and reach the \gls{ne}. The communication overhead will increase with the number of end devices. Furthermore, the works lack experiments with real applications and only approach the problem as a purely mathematical problem. 

\section{Learning-based Approaches}

In recent years, \gls{drl} has been used widely for decision-making problems. In edge computation offloading, the offloading strategy can also be modeled as a \gls{dnn}. Different offloading options are treated as the action space. Network conditions and robot onboard resources are considered as the state space. Since the learning-based approaches do not need a model of the system, they are more capable of dealing with dynamic changes in the network. 

\citeauthor*{Huang2019} \cite{Huang2019} train a \gls{dqn} agent to make offloading decisions for multi-users computation offloading. The reward function is formulated to reduce energy consumption and execution latency. \citeauthor*{Lu2020} \cite{Lu2020} take another step further with the \gls{dqn} agent by considering task partitioning for multi-user and multi-server computation offloading problems. The reward function is formulated to improve energy consumption, load balancing, and execution latency. 

Several works from these approaches also demonstrate real application performance. \citeauthor*{Chinchali2019}  \cite{Chinchali2019} train a \gls{a2c} agent to perform computation offloading with face recognition algorithm. The agent shows improved performance in accuracy and execution time compared with naive offloading strategies, such as only offloading to the edge, only computing locally, and randomly offloading. Moreover, \citeauthor*{Penmetcha2021} \cite{Penmetcha2021} offload navigation tasks to the cloud. The trained \gls{dqn} agent is capable of making offloading decisions by considering the data size of the task. Finally, \citeauthor*{Ruggeri2022} \cite{Ruggeri2022} trains a \gls{dqn} agent to make offloading decisions for scene understanding, which is a safety risk evaluation task. The safety risk is reduced compared to naive offloading strategies. 

Learning-based approaches show promising results in dealing with dynamic network changes and can make offloading decisions in a distributed manner. However, the computation of the \gls{dnn} also causes additional execution latency and increases the computation workload of the robot's onboard system. For time-sensitive tasks, such as object detection, the additional execution latency can cause the performance to deteriorate.

In the end, it is also worth noticing that a plethora of literature has already investigated the mathematical problem of computation offloading and various approaches have been proposed to tackle the problem. However, few investigate the influence of different strategies on the robotic system metrics in an experimental way. Therefore, this thesis intends to investigate this problem by implementing an offloading framework and conducting experiments on the influence of different offloading strategies. 