% In total max. 1 Page!
\RAIstudentthesisAbstract{%
%
% Abstract English:
% To achieve autonomy, \gls{amr} need to run different computation and memory intensive algorithms, such as perception, \ac{slam}, path planning, and learning. However, an \ac{amr}'s on-board computational and energy resources are limited by space restrictions and battery lifetime. With the emergence of edge computing, \acp{amr} have the opportunity to offload costly computation tasks to the edge, while only a small portion of the computation remains on the on-board system. However, depending on the application scenarios, offloading certain tasks from the \ac{amr} to the edge at all times may not be possible, beneficial or even feasible due to the network's latency, dynamic network changes, and resource availability. On the other hand, offloading computational workloads to the edge could impact the robot's safety, availability as well as on-board resources. The exact influence on the robotic system will depend on the chosen offloading strategy. Therefore, this thesis uses 2D object detection as an offloading task and investigates the performance and the latency, as well as the CPU usage and power consumption of the robot's on-board system under diffferent offloading strategies. 
% Furthermore, this thesis implements a decision-making offloading strategy based on the network conditions and robot's on-board system states to minimize the latency, but is also subjected to restrictions of the power consumption and CPU usage.

AMRs run various computationally intensive algorithms on their onboard systems, such as SLAM, perception, navigation, and path planning. However, their onboard systems usually have limited computation and energy resources. To investigate the effects of different offloading strategies, an offloading framework is designed and implemented using ROS2 for an object detection task using YOLOv5. Then, baseline offloading strategies are implemented and evaluated on defined metrics with experiments. After evaluating the baseline strategies, a dynamic offloading strategy that makes offloading decisions according to the states of the system is developed and implemented. In the end, an experiment on an actual robotic system using both Ethernet and Wi-Fi connections is carried out to evaluate the dynamic offloading strategy against the baselines. The results show that the dynamic offloading strategy not only improves the average precision of the object detection task but also reduces the resource usage of the onboard system compared to the baseline strategies. 
}{%
%
% Zusammenfassung Deutsch:
AMRs führen verschiedene rechenaufwändige Algorithmen auf ihren Onboard-Systemen aus, wie z.B. SLAM, Wahrnehmung, Navigation und Pfadplanung. Ihre Onboard-Systeme verfügen jedoch in der Regel über begrenzte Rechen- und Energieressourcen. Um die Auswirkungen verschiedener Offloading-Strategien zu untersuchen, wurde ein Offloading-Framework für eine Objekterkennungsaufgabe mit YOLOv5 entworfen und mit ROS2 implementiert. Dann wurden Baseline-Offloading-Strategien implementiert und anhand definierter Metriken mit Experimenten bewertet. Nach der Bewertung der Baseline-Strategien wurde eine dynamische Offloading-Strategie entwickelt und implementiert, die Offloading-Entscheidungen gemäß dem Zustand des Systems trifft. Schließlich wurde ein Experiment an einem Robotersystem mit Ethernet- und Wi-Fi-Verbindungen durchgeführt, um die dynamische Offloading-Strategie mit den Baseline-Strategien zu vergleichen. Die Ergebnisse zeigen, dass die dynamische Offloading-Strategie nicht nur die durchschnittliche Präzision der Objekterkennungsaufgabe verbessert, sondern auch den Ressourcenverbrauch des Onboard-Systems im Vergleich zu den Baseline-Strategien reduziert.
}%
%
%
